\documentclass[a4paper]{article}

\usepackage{amssymb}
\usepackage{euscript}
\usepackage{graphicx}
\graphicspath{{pics/}}
\usepackage{color}
\usepackage{epsfig}

\renewcommand{\arraystretch}{1.5}
\renewcommand{\tabcolsep}{1mm}

\begin{document}

\title{$^3$He library}
\date{\today}
\author{\tt vladislav.zavyalov@aalto.fi}
\maketitle

\subsection*{Usage}

This library provides constant and functions, related with $^3$He properties.

\noindent Supported Languages:

{\bf Fortran.}
You can use {\tt he3.fh} include file and {\tt libhe3} library in
fortran programs.

{\bf C.}
You can use {\tt he3.h} include file and {\tt libhe3} library in C programs.
Lowercase names with underscore should be used (like {\tt he3\_pf\_}).

{\bf Matlab, Octave.}
To compile {\tt mex} files you should run {\tt compile.m} script in the {\tt matlab}
folder. Lowercase names should be used.
Functions which take 1 argument can be used with array or matrix argument.
Functions which take 2 arguments can work either with two equal size matrices
or with number + matrix or matrix + number.

\subsection*{Location}
\begin{itemize}
\item[\bf GIT]  https://github.com/slazav/he3lib
\item[\bf ROTA] /rota/programs/src/he3lib -- old, ``stable'' version.\\
/home/slazav/he3lib/lib
\end{itemize}

%%%%%%%%%%%%%%%%%%%%%%%%%%%%%%%%%%%%%%%%%%%%%%%%%%%%%%%%%%%%%%%%%%%%%%
\subsection*{General}

\medskip
\begin{tabular}{lp{9cm}}
\tt he3\_amass     & $^3$He atom mass, $m_3 = 5.0079 \cdot 10^{-24}$~[g]\\
\tt he3\_mmass     & $^3$He molar mass, $\mu = 3.016$~[g/mol]\\
\tt he3\_gyro      & $^3$He gyromagnetic ratio $\gamma = 20378$~[(G s)$^{-1}$],\newline
                     (absolute value!)\\
\tt const\_na      & Avogadro constant $N_A = 6.02214129 \cdot 10^{23}$~[1/mol]\\
\tt const\_kb      & Boltsman constant $k_B = 1.3806488 \cdot 10^{-16}$~[sgs]\\
\tt const\_r       & R-gas constant $R = 8.314472 \cdot 10^{7}$~[sgs]\\
\tt const\_h       & Plank constant $h = 6.62606957 \cdot 10^{27}$~[sgs]\\
\tt const\_hbar    & $\hbar = h/2\pi = 1.054571726 \cdot 10^{27}$~[sgs]\\
\tt const\_pi      & $\pi = 3.1415926535897932$\\
\end{tabular}
\medskip

\noindent TODO: change $\gamma$ to negative value?

\eject
%%%%%%%%%%%%%%%%%%%%%%%%%%%%%%%%%%%%%%%%%%%%%%%%%%%%%%%%%%%%%%%%%%%%%%
\subsection*{Phase diagram}

Vapor pressure and critical point are from the {1962~$^3$He scale of
temperatures}. Melting curve at 0.9 -- 250~mK, minimum and triple points
on melting curve are from {PLTS-2000 temperature scale}. Melting curve
for higher temperatures is from {Osborne,Abraham,Weinstock-1951} and
{Mills,Grilly-1955}. Superfluid transition, AB-transition and A-B-Normal
triple point are from {Graywall-86}; values are corrected to fit
PLTS2000 temperature scale. See more information in the source file {\tt
he3\_phase.f}.

\medskip
\begin{tabular}{lp{9cm}}
\tt he3\_pvap(T)  & Vapor pressure [bar] vs temperature~[K],\newline
                    $ T = 0.2 - 3.324$~K \\
\tt he3\_pcr      & Gas-liquid critical point pressure, $78.111\cdot 10^{-3}$~[bar] \\
\tt he3\_tcr      & Gas-liquid critical point temperature, 3.324~[K] \\
\tt he3\_pmelt(T) & Melting pressure [bar] vs temperature~[K],\newline
                    $ T = 0.0009 - 31$~K\\
\tt he3\_pmelt\_gr(T) & Same, but using Greywall-86 temperature scale at
                    0.9--250~mK rather then PLTS2000\\
\tt he3\_psmin    & Melting curve minimum pressure, $29.3113$~[bar]\\
\tt he3\_tsmin    & Melting curve minimum temperature, $0.31524$~[K]\\
\tt he3\_ta       & Superfluid trans. at melting curve, temp., 2.444~[mK]\\
\tt he3\_pa       & Superfluid trans. at melting curve, pressure, 34.3407~[bar]\\
\tt he3\_tb       & A-B trans. at melting curve, temp., 1.1896~[mK]\\
\tt he3\_pb       & A-B trans. at melting curve, pressure, 34.3609~[bar]\\
\tt he3\_tneel    & Neel transition at melting curve, temp., 0.902~[mK]\\
\tt he3\_pneel    & Neel transition at melting curve, pressure, 34.3934~[bar]\\
\tt he3\_tc(P)    & Superfluid transition temperature [mK] vs pressure~[bar],\newline
                    $ P = 0 - 34.358 $~bar\\
\tt he3\_tab(P)   & A-B transition temperature [mK] vs pressure [bar],\newline
                    $ P = 0 - 34.3609 $~bar, below 21.22 bar is equal to {\tt he3\_tc}\\
\tt he3\_tabn     & A-B-Normal point temperature, 2.2311~[mK]\\
\tt he3\_pabn     & A-B-Normal point pressure, 21.22~[bar]\\
\end{tabular}
\medskip

\eject

\begin{figure}[h]
\includegraphics[scale=.6]{phase.eps}
\end{figure}

{\small\input{pics/phase.tex}}

\eject
%%%%%%%%%%%%%%%%%%%%%%%%%%%%%%%%%%%%%%%%%%%%%%%%%%%%%%%%%%%%%%%%%%%%%%
\subsection*{Fermi-liquid parameters}

Fermi-liquid parameters according to {Wheatley-75} table. Values are
calculated using experimental data for molar volume, speed of sound,
magnetic temperature $T^\star$, collected by {Wheatley-75} and heat
capacity ($\gamma_f=C/RT$) measured by {Graywall-86}. Argument is
pressure, [bar] in the range 0 -- 34.4. See more information in the
source file {\tt he3\_fermi.f}.

Measured values:

\medskip
\begin{tabular}{lp{9cm}}
\tt he3\_vm(P)     & Molar volume $v_m$,~[cm$^3$/mol]\\
\tt he3\_c1(P)     & First sound velocity, $c_1$,~[cm/s]\\
\tt he3\_gammaf(P) & R-gas constant $\gamma_f = C/RT$,~[1/(K mol)]\\
\tt he3\_tmag(P)   & Magnetic temperature $T^\star$,~[K]\newline\\
\end{tabular}
\medskip

Derived values:

\medskip
\begin{tabular}{lp{9cm}}
\tt he3\_rho(P)    & Density $\rho = \mu/v_m$,~[g/cm$^3$]\\
\tt he3\_2n0(P)    & $\displaystyle 2N(0) = \frac{\gamma_f}{v_m}
                     \ \frac{3 N_A}{k_B \pi^2}$\\
\tt he3\_pf(P)     & $\displaystyle p_F = h \left(\frac{3}{8\pi}
                     \ \frac{N_A}{v_m}\right)^{1/3}$,~[g cm/s]\\
\tt he3\_meff(P)   & $\displaystyle m^\star = \frac{h^3}{8\pi}
                     \ \frac{2N(0)}{p_F}$,~[g]\\
\tt he3\_mm(P)     & $m^\star/m_3$\\
\tt he3\_vf(P)     & $\displaystyle v_F = p_F/m^\star$,~[cm/s]\\
\tt he3\_chi\_n(P) & Normal phase susceptibility $\chi_{N}$.\newline
                   Without Fermi-liquid corrections (?)\\
\tt he3\_f0s(P)    &$\displaystyle F_0^s = 3\ m^\star m_3\ c_1^2 / p_F^2 - 1$\\
\tt he3\_f1s(P)    &$\displaystyle F_1^s = 3(m^\star/m_3 - 1)$\\
\tt he3\_f0a(P)    &$\displaystyle F_0^a = Z_0/4 = 3 k_B T^\star m^\star / p_F^2 - 1$\\
\tt he3\_f1a(P)    &$\displaystyle F_1^a$\\


\tt he3\_a(P)      &Average atomic spacing,
                    $\displaystyle a=(v_m/N_A)^{1/3}$,~\AA\\
\tt he3\_gdk(P)    &Average dipolar coupling enegy,
                    $\displaystyle g_d/k_B = \frac{2\pi\gamma^2\hbar^2}{3 v_m k_B}$,~[K]\\
\tt he3\_tfeff(P)  &Effective fermi temperature,
                    $\displaystyle T_{F_{eff}} = \frac{\pi^2}{2\gamma_f}$,~[K]\\
\tt he3\_lscatt(P) &Scattering factor $\lambda$\\
\end{tabular}
\medskip

\noindent TODO:\\
How to calculate $F_1^a$ from experimantal data?\\
Compare/update $F_0^a$ using Halperin-92.\\
Check $g_d$. In the texture library it is completely different!\\
Add $F_2^a$. Halperin-95\\
Fix susceptibility!

\eject
{\small\input{pics/fermi.tex}}

\eject
%%%%%%%%%%%%%%%%%%%%%%%%%%%%%%%%%%%%%%%%%%%%%%%%%%%%%%%%%%%%%%%%%%%%%%
\subsection*{Energy gap}

BCS energy gap + trivial strong coupling correction + some values
derived from energy gap. See more information in the
source file {\tt he3\_gap.f}.

\medskip
\begin{tabular}{lp{9cm}}
\tt he3\_bcsgap(ttc)    & BCS gap for 3He-B: $\Delta_{\rm BCS}/ k_B T_c$\\
\tt he3\_trivgap(ttc,P) & Trivial strong-coupling correction to the BCS gap\\
\tt he3\_yosida(ttc, gap, n) & Yosida functions $Y_n(\Delta, T/T_c) =$\newline
$\displaystyle
= \frac{1}{2T/T_c}\ \int_0^\infty
  \left(\frac{\xi}{\sqrt{\xi^2+\Delta^2}}\right)^n
  \left(\cosh \frac{\sqrt{\xi^2+\Delta^2}}{2T/T_c}\right)^{-2}
  d\xi
$\\

\tt he3\_chi\_b(ttc, gap)    &B-phase susceptibility\newline
                             $\displaystyle \chi_B = \chi_N \frac{(1+F_0^a)\chi_0}{1+F_0^a\chi_0},
                             \qquad \chi_0 = 2/3 + Y_0/3$\\
\tt he3\_nu\_b(ttc, gap)     &B-phase Leggett frequency\newline
                             $\displaystyle \nu_B =\frac{1}{2\pi}\sqrt{\frac{3\pi}{2\chi}}
                             \ \frac{\gamma^2\hbar}{2}\ N(0)\ \Delta \log\frac{e_f}{\Delta}  $\\
%\tt he3\_z3(ttc, gap)        &\\
%\tt he3\_z5(ttc, gap)        &\\
%\tt he3\_z7(ttc, gap)        &\\
%\tt he3\_yosida0(ttc, gap)   & Yosida function $Y_0$\\
\end{tabular}
\medskip

\noindent TODO:\\
Look at calculation of {\tt bcsgap}.\\
Do we need {\tt z3,z5,z7} from texture code? Probably
these values can be derived from {\tt yosida}.\\


\begin{figure}[h]
\includegraphics[scale=.6]{gap.eps}
\end{figure}

%%%%%%%%%%%%%%%%%%%%%%%%%%%%%%%%%%%%%%%%%%%%%%%%%%%%%%%%%%%%%%%%%%%%%%
\end{document}

