\documentclass[a4paper]{article}

\usepackage{amssymb}
\usepackage{euscript}
\usepackage{graphicx}
\graphicspath{{pics/}}
\usepackage{color}
\usepackage{epsfig}
\usepackage{fullpage}

\renewcommand{\arraystretch}{1.5}
\renewcommand{\tabcolsep}{1mm}

\begin{document}

\title{$^3$He library}
\date{\today}
\author{\tt vladislav.zavyalov@aalto.fi}
\maketitle

\subsection*{Usage}

This library provides constant and functions for various $^3$He properties.
Supported Languages:

\begin{itemize}
\item{Fortran-77.}
You should use {\tt he3.fh} include file and {\tt libhe3} library for
fortran programs. All functions and arguments are {\tt real*8}.
Use {\tt -fno-range-check} compiler flag to allow {\tt NaN} values.

\item{Fortran-90.}
You should include he3.f90h instead.

\item{\bf C.} You should use {\tt he3.h} include file and {\tt libhe3} library
in C programs. Lowercase names with underscore should be used (like {\tt
he3\_pf\_}). Type of arguments is always {\tt double*}, type of returned
value is {\tt double}.

\item{\bf Matlab, Octave.} {\tt mex} files are located in the {\tt matlab}
folder. Lowercase names should be used. In function arguments you can
mix numbers with arrays or matrices of the same size. Example:
\begin{verbatim}
  ttc=0:0.1:1; p=10;
  I = he3_nu_b(p, ttc);
\end{verbatim}
To build {\tt mex} files you can use {\tt make octave}, {\tt make matlab}
or {\tt make matlab64} in the {\tt matlab} folder. Fixed rpath is used
for the library. For building this documentation octave files are needed.


\item{\bf Command line.} You can use all the functions via a command line interface.
To get a list of all functions run the program {\tt he3} without arguments:
\begin{verbatim}
> he3
\end{verbatim}
To get information about specific function run the program with the
function name as an argument:
\begin{verbatim}
> he3 he3_nu_b
\end{verbatim}
You can use matlab-style vectors (value1:step:value2) to get a table
of values (if step is missing then 20 points will be printed):
\begin{verbatim}
> he3 he3_nu_b 0.1:0.1:1 0
\end{verbatim}
will give you Leggett frequency vs temperature at zero pressure.

\end{itemize}

\subsection*{Location}
\begin{itemize}
\item[GIT]  https://github.com/slazav/he3lib
\item[ROTA]
/home/slazav/he3lib/lib\\
/rota/programs/src/he3lib -- old, ``stable'' version.\\
Use ``{\tt addpath /rota/programs/src/he3lib/matlab}'' in Matlab.
\end{itemize}


%%%%%%%%%%%%%%%%%%%%%%%%%%%%%%%%%%%%%%%%%%%%%%%%%%%%%%%%%%%%%%%%%%%%%%
\subsection*{General}

\medskip
\noindent\begin{tabular}{lp{11cm}}
\tt const\_na      & Avogadro constant $N_A = 6.02214129 \cdot 10^{23}$~[1/mol]\\
\tt const\_kb      & Boltsman constant $k_B = 1.3806488 \cdot 10^{-16}$~[erg/K]\\
\tt const\_r       & R-gas constant $R = 8.314472 \cdot 10^{7}$~[sgs]\\
\tt const\_h       & Plank constant $h = 6.62606957 \cdot 10^{27}$~[sgs]\\
\tt const\_hbar    & $\hbar = h/2\pi = 1.054571726 \cdot 10^{27}$~[sgs]\\
\tt const\_pi      & $\pi = 3.1415926535897932$\\
\tt const\_2pi     & $2\pi$\\
\tt he3\_amass     & $^3$He atom mass, $m_3 = 5.0079 \cdot 10^{-24}$~[g]\\
\tt he3\_mmass     & $^3$He molar mass, $\mu = N_A m_3 = 3.0158$~[g/mol]\\
\tt he3\_gyro      & $^3$He gyromagnetic ratio $\gamma = 20378$~[(G s)$^{-1}$],\\
\end{tabular}
\medskip

\eject
%%%%%%%%%%%%%%%%%%%%%%%%%%%%%%%%%%%%%%%%%%%%%%%%%%%%%%%%%%%%%%%%%%%%%%
\subsection*{Phase diagram}

Vapor pressure and critical point are from the {1962~$^3$He scale of
temperatures}. Melting curve $P_m(T)$ at 0.9 -- 250~mK, $T_c(P)$,
$T_{AB}(P)$ and all tricritical points are from {Greywall-86 temperature
scale}. Melting curve minimum is from PLTS-2000 temperature scale.
Melting curve for higher temperatures is from {Osborne, Abraham, Weinstock-1951} and {Mills, Grilly-1955} with
interpolation in the regions 0.25--0.5K and 1.5--2.0K. PLTS-2000 melting
curve and critical points on it are also available. See more information
in the source file {\tt he3\_phase.f}.

\medskip
\noindent\begin{tabular}{lp{11cm}}
\tt he3\_pvap(T)  & Vapor pressure [bar] vs temperature~[K],\newline
                    $ T = 0.2 - 3.324$~K \\
\tt he3\_pcr      & Gas-liquid critical point pressure, 1.16317~[bar] \\
\tt he3\_tcr      & Gas-liquid critical point temperature, 3.324~[K] \\
\tt he3\_pmelt(T) & Melting pressure [bar] vs temperature~[K],\newline
                    $ T = 0.0009 - 31$~K\\
\tt he3\_pm       & Melting curve minimum pressure, $29.3113$~[bar]\\
\tt he3\_tm       & Melting curve minimum temperature, $0.31524$~[K]\\
\tt he3\_pa       & Superfluid trans. at melting curve, pressure, 34.3380~[bar]\\
\tt he3\_ta       & Superfluid trans. at melting curve, temp., 2.491~[mK]\\
\tt he3\_pb       & A-B trans. at melting curve, pressure, 34.3580~[bar]\\
\tt he3\_tb       & A-B trans. at melting curve, temp., 1.932~[mK]\\
\tt he3\_ps       & Neel transition at melting curve, pressure, 34.3905~[bar]\\
\tt he3\_ts       & Neel transition at melting curve, temp., 0.9291~[mK]\\
\tt he3\_tc(P)    & Superfluid transition temperature [mK] vs pressure~[bar],\newline
                    $ P = 0 - 34.358 $~bar\\
\tt he3\_tab(P)   & A-B transition temperature [mK] vs pressure [bar],\newline
                    $ P = 0 - 34.3609 $~bar, below 21.22 bar is equal to {\tt he3\_tc}\\
\tt he3\_tabn     & A-B-Normal point temperature, 2.2311~[mK]\\
\tt he3\_pabn     & A-B-Normal point pressure, 21.22~[bar]\\
\end{tabular}
\medskip

PLTS-2000 temperature scale:

\medskip
\noindent\begin{tabular}{lp{11cm}}
\tt he3\_pmelt\_plts(T) & PLTS-2000 melting presure [bar] vs temperature~[K].\\
\tt he3\_pa\_plts       & Superfluid trans. at melting curve, pressure, 34.3407~[bar]\\
\tt he3\_ta\_plts       & Superfluid trans. at melting curve, temp., 2.444~[mK]\\
\tt he3\_pb\_plts       & A-B trans. at melting curve, pressure, 34.3609~[bar]\\
\tt he3\_tb\_plts       & A-B trans. at melting curve, temp., 1.896~[mK]\\
\tt he3\_ps\_plts       & Neel transition at melting curve, pressure, 34.3934~[bar]\\
\tt he3\_ts\_plts       & Neel transition at melting curve, temp., 0.902~[mK]\\
\end{tabular}
\medskip

\begin{figure}[p]
\includegraphics[scale=1.3]{phase1.eps}\\
\includegraphics[scale=1.3]{phase2.eps}
\end{figure}

\eject
%%%%%%%%%%%%%%%%%%%%%%%%%%%%%%%%%%%%%%%%%%%%%%%%%%%%%%%%%%%%%%%%%%%%%%
\subsection*{Fermi-liquid parameters}

Fermi-liquid parameters according to {Wheatley-75} table. Values are
calculated using experimental data for molar volume, speed of sound,
magnetic temperature $T^\star$, collected by {Wheatley-75} and heat
capacity ($\gamma_f=C/RT$) measured by {Graywall-86}. Argument is
pressure, [bar] in the range 0 -- 34.4. See more information in the
source file {\tt he3\_fermi.f}.

\medskip
Measured values:

\medskip
\noindent\begin{tabular}{lp{11cm}}
\tt he3\_vm(P)     & Molar volume $v_m$,~[cm$^3$/mol]\\
\tt he3\_c1(P)     & First sound velocity, $c_1$,~[cm/s]\\
\tt he3\_gammaf(P) & R-gas constant $\gamma_f = C/RT$,~[1/(K mol)]\\
\tt he3\_tmag(P)   & Magnetic temperature $T^\star$,~[K]\newline\\
\end{tabular}

Derived values:

\medskip
\noindent\begin{tabular}{lp{11cm}}
\tt he3\_rho(P)    & Density $\rho = \mu/v_m$,~[g/cm$^3$]\\
\tt he3\_2n0(P)    & $\displaystyle 2N(0) = \frac{\gamma_f}{v_m}
                     \ \frac{3 N_A}{k_B \pi^2}$,~[1/(erg cm$^3$)]\\[3mm]
\tt he3\_pf(P)     & $\displaystyle p_F = h \left(\frac{3}{8\pi}
                     \ \frac{N_A}{v_m}\right)^{1/3}$,~[g cm/s]\\[3mm]
\tt he3\_meff(P)   & $\displaystyle m^\star = \frac{h^3}{8\pi}
                     \ \frac{2N(0)}{p_F}$,~[g]\\[3mm]
\tt he3\_mm(P)     &$m^\star/m_3$\\
\tt he3\_vf(P)     &$\displaystyle v_F = p_F/m^\star$,~[cm/s]\\
\tt he3\_chi\_n(P) &$\displaystyle \chi_{N} =
                     2N(0) \frac{(\gamma\hbar)^2}{4(1 + F_0^a)}$.\\[3mm]
\tt he3\_f0s(P)    &$\displaystyle F_0^s = 3\ m^\star m_3\ c_1^2 / p_F^2 - 1$\\
\tt he3\_f1s(P)    &$\displaystyle F_1^s = 3(m^\star/m_3 - 1)$\\
\tt he3\_f0a(P)    &$\displaystyle F_0^a = Z_0/4 = 3 k_B T^\star m^\star / p_F^2 - 1$\\
\tt he3\_f1a(P)    &$\displaystyle F_1^a$\\

\tt he3\_a(P)      &Average atomic spacing,
                    $\displaystyle a=(v_m/N_A)^{1/3}$,~\AA\\
\tt he3\_gdk(P)    &Average dipolar coupling enegy,
                    $\displaystyle g_d/k_B = \frac{2\pi\gamma^2\hbar^2}{3 v_m k_B}$,~[K]\\
\tt he3\_tfeff(P)  &Effective Fermi temperature,
                    $\displaystyle T_{F_{eff}} = \frac{\pi^2}{2\gamma_f}$,~[K]\\
\end{tabular}
\medskip

\noindent TODO:\\
Recalculate $F_0^a$ $F_1^a$ from experimantal data (Now there are some extra
not-consistent data)!
(see Ramm-1970, Halperin-92 ...). Add $F_2^a$ (zero or from Halperin-95).
Check $g_d$.

%%%%%%%%%%%%%%%%%%%%%%%%%%%%%%%%%%%%%%%%%%%%%%%%%%%%%%%%%%%%%%%%%%%%%%
\eject
Pressure-dependent values:
\medskip

{\small\input{ptable.tex}}

\eject
%%%%%%%%%%%%%%%%%%%%%%%%%%%%%%%%%%%%%%%%%%%%%%%%%%%%%%%%%%%%%%%%%%%%%%
\subsection*{Energy gap}

BCS energy gap + trivial strong coupling correction + some values
derived from energy gap. See more information in the
source file {\tt he3\_gap.f}.

\medskip
\noindent\begin{tabular}{lp{11cm}}
\tt he3\_bcsgap(ttc)       & BCS gap for 3He-B in $T_c$ units, $\Delta_{\rm BCS}/ k_B T_c$\\
\tt he3\_bcsgap\_fast(ttc) & Einzel approximation for BCS gap (0.5\% accuracy, 70 times faster)\newline
                             {\small(Einzel-1991, f.68)}\\
\tt he3\_trivgap(ttc, P)   & Trivial strong-coupling correction to the BCS gap\\
\tt he3\_todogap(ttc, P)   & Gap based on Todoschenko's measurements:
                             linear interpolation in density between BCS value at zero bar and
                             measured value 1.99 at melting pressure + temperature behaviour
                             as in {\tt he3\_trivgap}\\
\tt he3\_gap(ttc, P)   &     Wrapper which is used everywhere in the lib (trivgap by default)\\

\tt he3\_yosida(ttc, gap, n) & Yosida functions $\displaystyle Y_n(T/T_c,\Delta) =
                               \int_{-\infty}^{\infty} \left(\frac{\xi_k}{E_k}\right)^n
                               \phi_k\ d\xi_k$\par
                               Low temperature limit:
                               $\displaystyle Y_n(T,\Delta) =
                               2\Gamma\left(\frac{n+1}{2}\right)
                               \left(\frac{T}{\Delta}\right)^{\frac{n-1}{2}}
                               \exp\left(-\frac{\Delta}{T}\right)$\par
                               Note: type of n parameter in fortran should be real*8 (this was done
                               for standard function handling). If n is integer, result is unpredictable!\\
\tt he3\_yosida\_par(ttc, gap)  & $Y_q^\parallel(T/T_c,\Delta)$\\
\tt he3\_yosida\_perp(ttc, gap) & $Y_q^\perp(T/T_c,\Delta)$\\

\tt he3\_z3(ttc, gap) &\\
\tt he3\_z5(ttc, gap) &\\
\tt he3\_z7(ttc, gap) &\\
\tt he3\_lambda(ttc, gap) &\\

\tt he3\_rho\_nb(ttc, p)     &B-phase normal component density:
                             $\displaystyle \frac{\rho_B^n}{\rho_{N}} =
                               \frac{(3 + F_1^s) Y_0}{3 + F_1^s Y_0}$\\[3mm]
\tt he3\_chi\_b(ttc, p)      &B-phase susceptibility:
                             $\displaystyle \frac{\chi_B}{\chi_N} =
                               \frac{(1+F_0^a)(2 + Y_0)}
                                     {3+F_0^a(2 + Y_0)}$\\
\end{tabular}
\medskip

\begin{figure}[h]
\includegraphics[scale=1]{gap1.eps}\\
\end{figure}
\eject

%%%%%%%%%%%%%%%%%%%%%%%%%%%%%%%%%%%%%%%%%%%%%%%%%%%%%%%%%%%%%%%%%%%%%%
\subsection*{Dipole energy and Legget frequency}

Values are given according to Thuneberg's paper JLTP 122, p657 (2001).
Temperature independent $g_d$ parameter is restored from experimental
data for $\Omega_B$, measured in transverse and longitudinal NMR.
See more information in the source file {\tt he3\_dipole.f}.

Dipolar energy:
$$
F_D = \lambda_D \int (R_{ii}R_{jj}+R_{ij}R_{ji})\ d^3r
\quad =\quad 4 \lambda_D \int \cos\theta(1+2\cos\theta)\ d^3r
$$
$$
\lambda_D = \Delta^2 g_d,
\qquad \Omega_B = \gamma \sqrt{15 \lambda_D/\chi_B}
$$

Note: In old papers (Leggett, ...) $g_D/5$ is used instead of $\lambda_G$.

\medskip
\noindent\begin{tabular}{lp{11cm}}
\tt he3\_gd(p) & $g_d$,~[1/(erg cm$^3$)]\\
\tt he3\_ld(p) & $\lambda_D = \Delta^2 g_d$,~[erg/cm$^3$]\\
\tt he3\_nu\_b(ttc, p) &B-phase Leggett frequency
                             $\displaystyle \nu_B = \frac{\gamma}{2\pi}
                               \sqrt{15 \Delta^2 g_d/\chi_B}$, Hz\\

\tt he3\_nu\_b1(ttc, p)  & Less accurate formula without using $g_d$, Hz\newline
                             $\displaystyle \nu_B =\frac{1}{2\pi}\sqrt{\frac{3\pi}{2\chi}}
                             \ \frac{\gamma^2\hbar}{2}\ N(0)\ \Delta \log\frac{e_f}{\Delta}$\\
\end{tabular}
\medskip

\begin{figure}[h]
\includegraphics[scale=1.1]{nu_b.eps}\\
\end{figure}

\eject
%%%%%%%%%%%%%%%%%%%%%%%%%%%%%%%%%%%%%%%%%%%%%%%%%%%%%%%%%%%%%%%%%%%%%%
\subsection*{Gradient energy and spin wave velocity}

Gradient energy coefficients and derived values, including spin wave
velocities. See more information in the source file {\tt he3\_grad.f}.

$$
F_\nabla = \frac12 \Delta^2 \left[
  K_1 (\nabla_j R_{ak})(\nabla_j R_{ak})
+ K_2 (\nabla_j R_{ak})(\nabla_k R_{aj})
+ K_3 (\nabla_j R_{aj})(\nabla_k R_{ak}) \right]
$$

%Values are given according to Thuneberg's paper JLTP 122, p657 (2001).
%High order Fermi liquid parameters ($F_2^a$, $F_3^a$) are set to 0.
%See more information in the source file {\tt he3\_grad.f} and in
%{\tt doc\_tech/egrad.pdf}.
%$$
%F_\nabla  = \int_V \left[
%  \lambda_{G1}\frac{\partial R_{\alpha i}}{\partial r_i}
%              \frac{\partial R_{\alpha j}}{\partial r_j} +
%  \lambda_{G2}\frac{\partial R_{\alpha j}}{\partial r_i}
%              \frac{\partial R_{\alpha j}}{\partial r_i}
%  \right]\ d^3r
%+ 2\lambda_{G2} \int_S s_k R_{\mu k} \frac{\partial R_{\mu j}}{\partial r_j}\ d^2r
%$$
%($s_k$ is directed inside the volume)

\medskip
\noindent\begin{tabular}{p{4cm}p{11cm}}

\tt he3\_grad\_K0(ttc, p)  & $K_1=K_2=K_3$ without fermi-liquid corrections, see VW7.23m\\

\tt he3\_grad\_c(ttc, p)\par
\tt he3\_grad\_delta(ttc, p) & $c$ and $\delta$ paramters calculated with fermi-liquid
                            corrections (Cross-1975), see VW7.25. These values are
                            used to calculate all other things:\par
$\displaystyle
c=-\frac{\rho_s}{10}
\ \frac{3+F_1^a}{3+F_1^s}
\ \frac{1}{1+F_1^a(5-3\rho_s/\rho)/15}
,\qquad
\delta = \frac{F_1^a \rho_s/\rho}{3+F_1^s(1-\rho_s/\rho)}
$\\[3mm]
\tt he3\_grad\_K12(ttc, p)\par
\tt he3\_grad\_K3(ttc, p)  &
  $\displaystyle K_1 = K_2 = -\frac{2}{\Delta^2}\left(\frac{\hbar}{2m}\right)^2 c,\qquad
                 K_3 = -\frac{2}{\Delta^2}\left(\frac{\hbar}{2m}\right)^2 (1+\delta)c $\\

\tt he3\_grad\_K(ttc, p)\par
\tt he3\_grad\_Kp(ttc, p)  & $K  = 2K_1 + K_2 + K_3,\qquad K' = K_2 + K_3$\\

\tt he3\_grad\_lg1(ttc, p)\par
\tt he3\_grad\_lg2(ttc, p)\par
\tt he3\_grad\_lsgb(ttc, p) & Thunebergs $\lambda_{G1}$, $\lambda_{G2}$ and $\lambda_{SG}^b$:\par
$
\displaystyle
\lambda_{G1} = \frac12\Delta^2 (K_2+K_3),\qquad
\lambda_{G2} = \frac12\Delta^2 K_1,\qquad
\lambda_{SG}^b = \Delta^2 K_2
$\\

\tt he3\_cpar(ttc, p)\par
\tt he3\_cperp(ttc, p)    & velocity of transverse spin waves parallel
                            and perpendicular to the~$l$ direction [cm/s]:\par
$\displaystyle
c_\parallel^2 = \frac{\gamma^2\Delta^2}{\chi_B} K,\qquad
c_\perp^2 = \frac{\gamma^2\Delta^2}{\chi_B}(K-K'/2),\qquad
c_\parallel^2/c_\perp^2\approx 4/3$\\

\tt he3\_clpar(ttc, p)\par
\tt he3\_clperp(ttc, p)   & same for longitudinal waves [cm/s]:\par
$\displaystyle
C_\parallel^2 = \frac{\gamma^2\Delta^2}{\chi_B}(K-K'),\qquad
C_\perp^2 = \frac{\gamma^2\Delta^2}{\chi_B} K,\qquad
C_\parallel^2/C_\perp^2\approx 1/2$\\

\end{tabular}
\medskip

\begin{figure}[h]
\includegraphics[scale=0.8]{text_gr.eps}\\
\end{figure}

\eject
%%%%%%%%%%%%%%%%%%%%%%%%%%%%%%%%%%%%%%%%%%%%%%%%%%%%%%%%%%%%%%%%%%%%%%
\subsection*{Other textural parameters}

Values are given according to Thuneberg's paper JLTP 122, p657 (2001).
High order Fermi liquid parameters ($F_2^a$, $F_3^a$) are set to 0.
See more information in the source file {\tt he3\_text.f}.

Bulk terms:
\begin{eqnarray*}
F_{DH} &=& -a
  \int_V ({\bf n} \cdot {\bf H})^2\ d^3r\\
F_{DV} &=& -\lambda_{DV}
  \int_V [{\bf n} \cdot ({\bf v_s}-{\bf v_n}) ]^2\ d^3r
\qquad {\ll F_{HV} \mbox{~at~} H\approx 200 \mbox{~Oe}}\\
F_{HV} &=& -\lambda_{HV}
  \int_V [{\bf H} \cdot R \cdot ({\bf v_s}-{\bf v_n}) ]^2\ d^3r\\
\end{eqnarray*}

Surface terms:
\begin{eqnarray*}
F_{SH} &=& -d
  \int_S [{\bf H} \cdot R \cdot {\bf s} ]^2\ d^2r\\
F_{SG} &=& -\lambda_{SG}
  \int_S s_j R_{\alpha j}
    \frac{\partial R_{\alpha i}}{\partial r_i}\ d^2r\\
\end{eqnarray*}

Vortex term:
\begin{eqnarray*}
F_{LH} &=& \frac{\lambda_{LH}}{2\Omega}
  \int_L |\omega_s| [{\bf H} \cdot R \cdot {\bf l} ]^2\ d^3r\\
\end{eqnarray*}

\hrule

\medskip
\noindent\begin{tabular}{lp{11cm}}

\tt he3\_text\_a(ttc, p) & $a = \displaystyle
\frac{5 g_d}{2}
\left[\frac{\hbar \gamma / 2}{1 + F_0^a (2+Y_0)/3}\right]^2
\left(5-3\frac{Z_3}{Z_5}\right)$, [erg/cm$^3$ G$^{-2}$] \\[5mm]

\tt he3\_text\_d(ttc, p) & $d$, [erg/(cm$^2$ G$^2$)] -- GL extrapolation!\\

\tt he3\_text\_ldv(ttc, p) & $\lambda_{DV} = \displaystyle
5 g_d
\left( \frac{m^* v_F}{1 + F_1^s Y_0/3} \right)^2
\left( 1 - \frac{3 Z_5}{2 Z_3} \right)$, [erg/cm$^3$ (cm/s)$^{-2}$]\\[5mm]

\tt he3\_text\_lhv(ttc, p) & $\lambda_{HV} = \displaystyle
\frac{\rho}{\Delta^2}
\frac{1 + F_1^s / 3}{(1 + F_1^s Y_0/3)^2}
\left( \frac{\hbar\gamma/2}{ 1 + F_0^a (2 + Y_0)/3)} \right)^2
\left( Z_3 - \frac{9}{10} Z_5 + \frac{9}{10}\ \frac{Z_5^2}{Z_3} -
  \frac32 Z_7 \right) $\\[5mm]

\tt he3\_text\_llh(ttc, p, omega) & $\lambda_{LH} = \displaystyle
 \frac{\hbar}{2m}\Omega \lambda_{HV}
 \left(\ln\frac{R}{r} - \frac34\right)$, [erg/(cm$^3$ G$^2$ s)] -- counterflow part only!\\[5mm]

\tt he3\_text\_lo(ttc, p, omega) & lambda/omega
   used in the texture library: $\displaystyle \frac{\lambda}{\Omega} =
   \frac{5\lambda_{LH}}{2a\Omega}$, (rad/s)$^{-1}$\\


\end{tabular}
\medskip

\hrule

\medskip
\noindent\begin{tabular}{lp{11cm}}

\tt he3\_text\_xih(ttc, p, h) & Magnetic length
  $\xi_H = \sqrt{65\lambda_{G2}/(8 a H^2)}$\\

\tt he3\_text\_xid(ttc, p) & Dipolar length (according to Hakonen-1989)
  $\xi_D = \sqrt{13\lambda_{G2}/12\lambda_D}$.\par
  Note: in Thuneberg's paper $\xi_D = \sqrt{\lambda_{G2}/\lambda_D}$ is used.\\

\tt he3\_text\_vd(ttc, p) & Dipolar velocity (according to Thuneberg-2001)
  $v_D = \sqrt{2a/5\lambda_{HV}}$\\


\end{tabular}
\medskip



\begin{figure}[p]
\includegraphics[scale=0.8]{text_pars.eps}\\
\end{figure}

\begin{figure}[p]
\includegraphics[scale=0.8]{text_pars2.eps}\\
\end{figure}

\eject
%%%%%%%%%%%%%%%%%%%%%%%%%%%%%%%%%%%%%%%%%%%%%%%%%%%%%%%%%%%%%%%%%%%%%%
\subsection*{Transport properties in the normal phase}

See more information in the source file {\tt he3\_transp\_n.f}.

\medskip

Crossections and scattering factors:

\medskip
\noindent\begin{tabular}{lp{11cm}}
\tt he3\_scatt\_w(P)   & Scattering crossection ${<}W{>}$ {\small(Einzel-1978, f.82)}\\
\tt he3\_scatt\_wi(P)  & ${<}W_I{>}$ {\small(Einzel-1978, f.82)}\\
\tt he3\_scatt\_wd(P)  & ${<}W_D{>}$ {\small(Einzel-1978, f.82)}\\
\tt he3\_scatt\_wl(P)  & ${<}W_L{>}$ {\small(Einzel-1978, f.71)}\\
\tt he3\_scatt\_g0(P)  & Scattering factor $\gamma_0 = <W_I>/<W>$ {\small(Einzel-1978, f.66)}\\
\tt he3\_scatt\_d0(P)  & Scattering factor $\delta_0 = <W_D>/<W>$ {\small(Einzel-1978, f.67)}\\
\tt he3\_scatt\_w0(P)  & $w_0 = 1 - 3/2 \gamma_0 + \delta_0$ {\small(Einzel-1978, f.79)}\\
\tt he3\_scatt\_l1a(P) & Scattering factor $\lambda_1^a$ used in spin diffusion
                         transport time. Can be neglected.
                         \newline {\small(Einzel-1978, f.74, p.350)}.
\end{tabular}
\medskip

\medskip
\noindent\begin{tabular}{lp{11cm}}
\tt he3\_tau\_n0(ttc,p) &Normal state quasiparticle lifetime at the Fermi level, s.\newline
                    $\displaystyle \tau_N(0,T) =
                    \tau_N(0,T_c) \left(\frac{T_c}{T}\right)^2
                    = \frac{32 E_F\hbar}{{<}W{>}(\pi k_B T)^2}$\newline
                    {\small(Einzel-1991, p.325)}\\
\tt he3\_tau\_n\_av(ttc,p) &Thermal average of normal state quasiparticle lifetime, s.\newline
                    $\displaystyle \bar\tau_N =
                    \left<\frac{\tau_N(0,T)}{1+(\xi_k/\pi T)^2}\right>
                    = \frac34 \tau_N(0,T)\quad$
                    {\small(Einzel-1991, f.4-5)}\\
\tt he3\_tau\_nd(ttc,p) & Spin diffusion transport time for a normal Fermi-liquid, s\newline
                    $\displaystyle \tau_{ND} = \tau_1 = \bar\tau_N/(1-\lambda_1^a)\quad$
                    {\small(Einzel-1991, p.328)}\\

\tt he3\_diffn\_hydr(ttc, p)     & Hydrodynamic spin diffusion in normal liquid, $D_0$, cm$^2$/s\newline
                              {\small (Einzel JLTP84 (1991) f.23)}\\
\tt he3\_diffn\_perp(ttc, p, nu0) & Perpendicular component of spin diffusion, reduced because of
                              the Leggett-Rice effect, $D_N^\perp$, cm$^2$/s\newline
                              {\small (Einzel JLTP84 (1991) f.22)}\\
\end{tabular}
\medskip

\noindent TODO: Temperature is measured in $T_c$ units which is strange for the normal phase.
Maybe it is better to use K of $T_F$ units here.

\eject
%%%%%%%%%%%%%%%%%%%%%%%%%%%%%%%%%%%%%%%%%%%%%%%%%%%%%%%%%%%%%%%%%%%%%%
\subsection*{Transport properties in the B phase}

Values for {\tt he3\_tau0, he3\_tau\_av, he3\_diff\_*} are extended to
$T>T_c$ region using normal phase functions. See more information in the
source file {\tt he3\_transp\_b.f} and in {\tt doc\_tech/spin\_diff.pdf}.

\medskip

Collision integral for Bogoliubov quasiparticles:

\medskip
\noindent\begin{tabular}{lp{11cm}}
\tt he3\_coll\_int(xi, ttc, gap, g0, d0) &
   Full temperature range approximation\newline
   {\small (Einzel, Wolfle, Hirschfeld, JLTP80 (1990), p.66)}\\
\tt he3\_coll\_int\_lt(xi, ttc, gap, g0, d0) &
   Collision integral for low temp (good for $< 0.7 T_c$)\newline
   {\small (Einzel, JLTP84 (1991), f.76)}\\
\tt he3\_coll\_int\_ht(xi, ttc, gap, g0, d0) &
   Collision integral for high temp\newline
   {\small (Einzel, JLTP84 (1991), f.75)}\\
\end{tabular}
\medskip

Bogoliubov quasiparticles lifetime and mean free path:

\medskip
\noindent\begin{tabular}{p{4cm}p{11cm}}
\tt he3\_tau0(ttc, p)      & Lifetime at the Fermi level, s.
                           $\displaystyle\quad 1/\tau(0) = \frac{I (0,T)}{\tau_N(0,T)},$\newline
                            {\small(Einzel-1991, f.74,76. Einzel-1990 f.A1)}\\

\tt he3\_tau0lt(ttc, p)    & $\displaystyle 1/\tau, T\rightarrow0$ limit (does not depend on energy).
                            {\small(Einzel-1978, f.79)}\\
\tt he3\_tau\_av(ttc, p)   & Thermal average lifetime, s.
                           $\displaystyle\quad 1/\bar\tau = \frac{1}{Y_0}
                           \int_{-\infty}^{\infty} \frac{\phi_k\ d\xi}{\tau}$\newline
                            {\small(Einzel-1991, f.77)}\\
\tt he3\_fpath(ttc, p)     & Mean free path, cm\newline
                             {\small (Einzel JLTP32 (1978) f.84)}\\

\tt he3\_tau\_dperp(ttc, p)\newline
\tt he3\_tau\_dpar(ttc, p)  & Spin diff. transport time, s.
                           \ $\tau_D^{\perp,\parallel} = \bar\tau
                            /(1-\lambda_1^a Y^{\perp,\parallel})$\newline
                            {\small(Einzel-1991, f.90,96)}\\
\end{tabular}
\medskip

Spin diffusion:

\medskip
\noindent\begin{tabular}{p{6cm}p{10cm}}
\tt he3\_diff\_hperp\_zz(ttc, p)\newline
\tt he3\_diff\_hpar\_zz(ttc, p)
& Spin diffusion in hydrodynamic limit ($\omega_L \tau \ll 1$),
  $D^{\perp,\parallel}_{zz}(\omega_L=0)$, cm$^2$/s. According to Einzel's
  paper (see f.105) nonhydrodynamic effects do not affect $D^\parallel$.
  In Mukharsky paper they do.\newline
  {(\small Einzel JLTP84 (1991) f.102)}\\

\tt he3\_diff\_perp\_xx(ttc, p, nu0)\newline
\tt he3\_diff\_perp\_zz(ttc, p, nu0)\newline
\tt he3\_diff\_perp\_xx\_im(ttc, p, nu0)\newline
\tt he3\_diff\_perp\_zz\_im(ttc, p, nu0)\newline
\tt he3\_diff\_par\_xx(ttc, p, nu0)\newline
\tt he3\_diff\_par\_zz(ttc, p, nu0)\newline
&Components of spin diffusion tenzor, cm$^2$/s\newline
{\small(Bunkov et al, 1990 f.3; Einzel, 1991, f.108;\newline
Markelov, Mukharsky, 1992, f.7,8)}\\
\end{tabular}
\medskip

Problem with $D^\parallel$. According to Einzel-1991 (see f.105)
nonhydrodynamic effects do not affect $D^\parallel$ and you should use
he3\_diff\_hpar\_zz for any frequency.
In Markelov-Mukharsky paper the result for $D^\parallel$ is completely
different (it is used in he3\_diff\_par\_xx, \_zz).
$D^\perp$ is the same in all papers.


\noindent TODO:\\
Exchange coupling strength is in the $F_1^a = 0$ limit. (small difference from normal phase values)\\
How does {\tt he3\_diff\_perp\_??\_im} extends to normal phase?\\

\begin{figure}[p]
\includegraphics[scale=1.1]{transp_tau.eps}\\
\includegraphics[scale=1.1]{transp_diff.eps}
\end{figure}

\begin{figure}[p]
\includegraphics[scale=0.75]{sdiff_per.eps}\\
\includegraphics[scale=0.75]{sdiff_par.eps}
\end{figure}

\eject

%%%%%%%%%%%%%%%%%%%%%%%%%%%%%%%%%%%%%%%%%%%%%%%%%%%%%%%%%%%%%%%%%%%%%%
\subsection*{Other}
See more information in the source file {\tt he3\_other.f}.

\medskip
\noindent\begin{tabular}{lp{11cm}}
\tt he3\_xigl(ttc, p) & Extrapolated GL coherence length, cm\\
\tt he3\_vneq(ttc, p, omega, r) & Equilibrium vortex number\\

\end{tabular}
\medskip

%%%%%%%%%%%%%%%%%%%%%%%%%%%%%%%%%%%%%%%%%%%%%%%%%%%%%%%%%%%%%%%%%%%%%%
\subsection*{Normal liquid beyond zero temperature limit}
See more information in the source file {\tt he3\_normal.f}.

\medskip
\noindent\begin{tabular}{lp{11cm}}
\tt he3\_cv\_n(t, p)  &Normal phase heat capacity $C_v$, {\small Greywall-1983}\\

\end{tabular}
\medskip

TODO: $V_m$

%%%%%%%%%%%%%%%%%%%%%%%%%%%%%%%%%%%%%%%%%%%%%%%%%%%%%%%%%%%%%%%%%%%%%%
\subsection*{ROTA-specific functions and constants}
See more information in the source files {\tt he3\_rota.f} and {\tt he3\_const.f}

\medskip
\noindent\begin{tabular}{lp{11cm}}
\tt rota\_c\_ns(t, i) &Nuclear stage heat capacity [J/K] vs T[K] and I[A]\newline
   $C_{ns} = 9.66 \cdot 10^{-5}\ {\rm [J/T^2]} \ (0.113\ {\rm [T/A]}\  I/T)^2$\\

\tt rota\_fork\_cal(w, p, n) &Calibration of fork $n$, $T/T_c$, vs fork width $w$~[Hz] and pressure $P$~[bar]\newline
   $\displaystyle w = a_n
     \exp\left(-\frac{\Delta_{\rm WC+}(P, T/T_c)}{T/T_c}\right),$\newline
   $a = \alpha_n \ p_F^4(P),$ where $\alpha_n$ is a geometrical factor\newline
   $N=1:\qquad $ Fork K, calibration 30.4.2010, 29 bar, a = 11700\newline
   $N=2:\qquad $ Fork E, calibration 30.4.2010, 29 bar, a = 17543\\

\tt rota\_rcell     & cell radius\\
\tt rota\_nmra      & field/current in nmrA solenoid [G/A]\\
\tt rota\_nmrb      & field/current in nmrB solenoid [G/A]\\
\tt rota\_nmra\_q(f0) & Q value of the nmrA spectrometer vs frequency\\
\tt rota\_nmra\_f(n)  & frequencies of nmrA spectrometer,kHz for n=1..8\\
\tt rota\_hmina\_r  & effective radius of the HminA coil [cm]\\
\tt rota\_hmina     & field/current in the center of HminA coil [G/A]\\
\tt rota\_hmina\_mr & quadratic radial term of the HminA field, [G/A/cm$^2$]\\
\tt rota\_hmina\_i0 & effectve HminA coil current caused by NMR field distortion, divided by NMR freq [G/Hz]\\
\tt rota\_rrda      & radiation damping constant $R_{RD}$ for the nmrA spectrometer\\
\end{tabular}

\eject
%%%%%%%%%%%%%%%%%%%%%%%%%%%%%%%%%%%%%%%%%%%%%%%%%%%%%%%%%%%%%%%%%%%%%%
\subsection*{Q-balls in the zero temperature limit}
See more information in the source file {\tt he3\_qball.f}

\medskip
\noindent\begin{tabular}{p{4cm}p{11cm}}
\tt qball\_nu\_b(P) & Leggett frequency (measurements, same as he3\_nu\_b(0,p) ) [Hz]\\
\tt qball\_cpar(P)\par
\tt qball\_cper(P)  & Measured $c_\parallel$, $c_\perp$, [cm/s]\\
\tt qball\_lg1(P)\par
\tt qball\_lg2(P)    & $\lambda_{G1}$, $\lambda_{G2}$ calculated
                      using measured spin wave velocities.\\
\tt qball\_text\_a(P)      &  measured textural parameter $a$ [erg/cm$^3$ 1/G$^2$]\\
\tt qball\_text\_d(P)      &  measured textural parameter $d$ [erg/(cm$^2$ G$^2$)]\\
\tt qball\_text\_lsg(P)    &  measured textural parameter $\lambda_{SG}$\\

\tt qball\_trd(P,f0,fr,fz) & $\tau_{RD}$ for the magnon condensate with given radial and axial frequencies (rota-specific) [s]\\
\end{tabular}

A model of the zero-temperature, zero-rotation condensate, based on spectra measurements (rota-specific):

\medskip
\noindent\begin{tabular}{p{4cm}p{11cm}}
\tt qball\_dbetan(P, f0)  & Derivative of the textural angle $\beta_N$ in the center of the cell, [rad/cm].\\
\tt qball\_fz0(P,f0,Imin)\par
\tt qball\_fr0(P,f0,Imin)  & $1/2$ of the distance between visible axial and radial levels, [Hz]\\
\tt qball\_az0(P,f0,Imin)\par
\tt qball\_ar0(P,f0,Imin)  & Axial and radial sizes of the magnon condensate, [cm]\\
\tt qball\_trd0(P,f0,Imin) & $\tau_{RD}$ for the magnon condensate, [s]\\
\end{tabular}

\begin{figure}[h]
\includegraphics[scale=1.1]{qball.eps}\\
\end{figure}


%%%%%%%%%%%%%%%%%%%%%%%%%%%%%%%%%%%%%%%%%%%%%%%%%%%%%%%%%%%%%%%%%%%%%%


\end{document}

